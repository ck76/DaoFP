\documentclass[DaoFP]{subfiles}
\begin{document}
 \setcounter{chapter}{4}

 \chapter{Product Types\\积类型}

 我们可以使用和类型枚举给定类型的可能值,但这种编码可能是低效的。我们需要十个构造器才能编码从零到九的数字。
 \begin{haskell}
  data Digit = Zero | One | Two | Three | ... | Nine
 \end{haskell}
 但是,如果我们将两个数字组合成一个数据结构,即一个两位数的十进制数,我们就能编码一百个数字。正如老子所说,只需四位数字就可以编码一万个数字。

 这种以这种方式组合两种类型的数据类型称为积,或\index{cartesian product}\emph{笛卡尔}积。它的定义特性是消去规则:有两个箭头从$a \times b$出发;一个叫“$\text{fst}$”,指向$a$,另一个叫“$\text{snd}$”,指向$b$。它们被称为(笛卡尔)\index{cartesian projections}\emph{投影}。它们让我们从积$a \times b$中检索$a$和$b$。

 \[
  \begin{tikzcd}
   & a \times b
   \arrow[dl,  "\text{fst}"]
   \arrow[dr,   "\text{snd}"']
   \\
   a && b
  \end{tikzcd}
 \]

 假设有人给了你一个积的元素,即一个从终端对象$1$到$a \times b$的箭头$h$。你可以通过组合轻松检索一对元素:一个给定$a$的元素
 \[x = \text{fst} \circ h \]
 和一个给定$b$的元素
 \[y = \text{snd} \circ h \]

 \[
  \begin{tikzcd}
   & 1
   \arrow[d, dashed, "h"]
   \arrow[ddl, bend right, "x"']
   \arrow[ddr, bend left, "y"]
   \\
   &a \times b
   \arrow[dl,  "\text{fst}"]
   \arrow[dr,   "\text{snd}"']
   \\
   a && b
  \end{tikzcd}
 \]

 实际上,给定一个从任意对象$c$到$a \times b$的箭头,我们可以通过组合定义一对箭头$f \colon c \to a$和$g \colon c \to b$。

 \[
  \begin{tikzcd}
   & c
   \arrow[d, dashed, "h"]
   \arrow[ddl, bend right, "f"']
   \arrow[ddr, bend left, "g"]
   \\
   &a \times b
   \arrow[dl,  "\text{fst}"]
   \arrow[dr,   "\text{snd}"']
   \\
   a && b
  \end{tikzcd}
 \]

 正如我们之前对和类型所做的那样,我们可以反过来使用这个图来\emph{定义}积类型:我们规定一个函数对$f$和$g$与从$c$到$a \times b$的\index{mapping in}\emph{映射}是一对一对应的关系。这是积的\emph{引入}规则。

 特别地,从终端对象的映射在Haskell中用于定义积类型。给定两个元素,\hask{a :: A}和\hask{b :: B},我们构造积

 \begin{haskell}
 (a, b) :: (A, B)
 \end{haskell}
 内置的积语法就是这样:一对括号和中间的逗号。它既适用于定义两个类型的积\hask{(A, B)},也适用于数据构造器\hask{(a, b)},它接受两个元素并将它们配对在一起。

 我们应该始终牢记编程的目的:将复杂问题分解为一系列简单的问题。在积的定义中,我们再次看到这一点。每当我们必须构造一个\emph{进入}积的映射时,我们将其分解为两个较小的任务,即构造一个函数对,每个函数映射到积的一个组成部分。这就像说,为了实现一个返回一对值的函数,足以实现两个函数,每个函数返回对中一个元素。

 \subsection{Logic\\逻辑}

 在逻辑中,积类型对应于逻辑合取。为了证明$A \times B$($A$和$B$),你需要同时提供$A$和$B$的证明。这些是指向$A$和$B$的箭头。消去规则表明,如果你有$A \times B$的证明,那么你自动获得$A$的证明(通过$\text{fst}$)和$B$的证明(通过$\text{snd}$)。

 \subsection{Tuples and Records\\元组和记录}

 正如老子所说,万物的积不过是一个具有万物投影的对象。

 我们可以使用元组表示法在Haskell中形成任意积。例如,三种类型的积写作\hask{(A, B, C)}。该类型的一个项可以由三个元素构造:\hask{(a, b, c)}。

 在数学家所称的“符号滥用”中,零个类型的积写作\hask{()},一个空元组,它恰好与终端对象或单位类型相同。这是因为积的行为非常像数字的乘法,终端对象扮演了“一”的角色。

 在Haskell中,与其为所有元组定义单独的投影,我们使用模式匹配语法。例如,要从三元组中提取第三个组件,我们可以这样写

 \begin{haskell}
  thrd :: (a, b, c) -> c
  thrd (_, _, c) = c
 \end{haskell}
 我们对想忽略的组件使用通配符。

 老子说,“命名是所有特殊事物的起源。”在编程中,如果不给它们命名,跟踪特定元组中组件的含义是困难的。记录语法允许我们为投影命名。这是以记录方式编写的积的定义:
 \begin{haskell}
  data Product a b = Pair { fst :: a, snd :: b }
 \end{haskell}
 \hask{Pair}是数据构造器,\hask{fst}和\hask{snd}是投影。

 这是如何声明和初始化特定对的:
 \begin{haskell}
  ic :: Product Int Char
  ic = Pair 10 'A'
 \end{haskell}

 \section{Cartesian Category\\笛卡尔范畴}

 在Haskell中,我们可以定义任意两种类型的积。具有所有积和终端对象的范畴称为\emph{笛卡尔}范畴。

 \subsection{Tuple Arithmetic\\元组算术}

 积所满足的恒等式可以使用映射性质导出。例如,要证明$a \times b \cong b \times a$,考虑以下两个图:

 \[
  \begin{tikzcd}
   &x
   \arrow[d, dashed, "h"]
   \arrow[ddl, bend right, "f"']
   \arrow[ddr, bend left, "g"]
   \\
   & a \times b
   \arrow[dl,  "\text{fst}"]
   \arrow[dr,   "\text{snd}"']
   \\
   a && b
  \end{tikzcd}
  \qquad
  \begin{tikzcd}
   &x
   \arrow[d, dashed, "h'"]
   \arrow[ddl, bend right, "g"']
   \arrow[ddr, bend left, "f"]
   \\
   & b \times a
   \arrow[dl,  "\text{fst}"]
   \arrow[dr,   "\text{snd}"']
   \\
   b && a
  \end{tikzcd}
 \]

 它们显示,对于任何对象$x$,到$a \times b$的箭头与到$b \times a$的箭头是一对一对应的。这是因为这些箭头中的每一个都由相同的$f$和$g$对确定。

 你可以检查自然性条件是否得到了满足,因为当你使用箭头$k \colon x' \to x$切换焦点时,所有起始于$x$的箭头都通过前组合$(- \circ k)$来转换。

 在Haskell中,这个同构可以实现为一个函数,它是其自身的逆:
 \begin{haskell}
  swap :: (a, b) -> (b, a)
  swap x = (snd x, fst x)
 \end{haskell}
 这是使用模式匹配编写的相同函数:
 \begin{haskell}
  swap (x, y) = (y, x)
 \end{haskell}

 重要的是要记住,积仅在“同构意义上”是对称的。这并不意味着交换对的顺序不会改变程序的行为。对称性意味着交换对的信息内容相同,但访问它需要进行修改。

 终端对象是积的单位,$1 \times a \cong a$。见证$1 \times a$和$a$之间同构的箭头称为\emph{左单位元}:
 \[ \lambda \colon 1 \times a \to a \]
 它可以实现为$\lambda = \text{snd}$。它的逆$\lambda^{-1}$定义为以下图中的唯一箭头:
 \[
  \begin{tikzcd}
   &a
   \arrow[d, dashed, "\lambda^{-1}"]
   \arrow[ddl, bend right, "!"']
   \arrow[ddr, bend left, "\text{id}"]
   \\
   & 1 \times a
   \arrow[dl,  "\text{fst}"]
   \arrow[dr,   "\text{snd}"']
   \\
   1 && a
  \end{tikzcd}
 \]
 从$a$到$1$的箭头称为\index{!}$!$(发音为,\emph{bang})。这确实表明
 \[\text{snd} \circ \lambda^{-1} = \text{id} \]
 我们仍然需要证明$\lambda^{-1}$是$\text{snd}$的左逆。考虑以下图:
 \[
  \begin{tikzcd}
   &1 \times a
   \arrow[d, dashed, "h"]
   \arrow[ddl, bend right, "!"']
   \arrow[ddr, bend left, "\text{snd}"]
   \\
   & 1 \times a
   \arrow[dl,  "\text{fst}"]
   \arrow[dr,   "\text{snd}"']
   \\
   1 && a
  \end{tikzcd}
 \]
 很明显,对于$h = id$来说,这个图是交换的。对于$h = \lambda^{-1}  \circ  \text{snd}$,这个图也是交换的,因为我们有:
 \[  \text{snd} \circ \lambda^{-1}  \circ  \text{snd} = \text{snd} \]
 由于$h$应该是唯一的,我们得出结论:
 \[ \lambda^{-1}  \circ  \text{snd} = id \]
 这种使用普遍构造的推理是相当标准的。

 以下是用Haskell编写的其他一些同构(没有证明逆)。这是结合律:
 \begin{haskell}
  assoc :: ((a, b), c) -> (a, (b, c))
  assoc ((a, b), c) = (a, (b, c))
 \end{haskell}
 这是右单位
 \begin{haskell}
  runit :: (a, ()) -> a
  runit (a, _) = a
 \end{haskell}

 这两个函数对应于\index{associator}\emph{结合子}
 \[ \alpha \colon (a \times b) \times c \to a \times (b \times c) \]
 和\index{unitor}\emph{右单位元}:
 \[ \rho \colon a \times 1 \to a \]

 \begin{exercise}
  证明左单位证明中的双射是自然的。提示,使用箭头$g \colon a \to b$来更改焦点。
 \end{exercise}

 \begin{exercise}
  构造一个箭头
  \[ h \colon b + a \times b \to (1 + a) \times b \]
  这个箭头是唯一的吗?

  提示:这是映射到积的映射,所以它由一对箭头给出。这些箭头,反过来,从和类型中映射出来,因此每一个都是由一对箭头给出的。

  提示:映射$b \to 1 + a$由$(\text{Left} \, \circ \, !)$给出
 \end{exercise}

 \begin{exercise}
  重做上一个练习,这次将$h$视为从和类型中映射\emph{出}。
 \end{exercise}

 \begin{exercise}
  实现一个Haskell函数\hask{maybeAB :: Either b (a, b) -> (Maybe a, b)}。这个函数是由其类型签名唯一定义的吗?还是有一些余地?
 \end{exercise}

 \subsection{Functoriality\\函子性}

 假设我们有箭头将$a$和$b$映射到某个$a'$和$b'$:
 \begin{align*}
  f &\colon a \to a' \\
  g &\colon b \to b'
 \end{align*}
 这些箭头与投影$\text{fst}$和$\text{snd}$的组合分别用于定义积之间的映射$h$:

 \[
  \begin{tikzcd}
   & a \times b
   \arrow[d, dashed, "h"]
   \arrow[dl,  "\text{fst}"']
   \arrow[dr,   "\text{snd}"]
   \\
   a
   \arrow[d, "f"']
   &a' \times b'
   \arrow[dl,  "\text{fst}"]
   \arrow[dr,   "\text{snd}"']
   & b
   \arrow[d, "g"]
   \\
   a' && b'
  \end{tikzcd}
 \]
 这个图的简写符号是:
 \[ a \times b \xrightarrow{f \times g} a' \times b' \]

 这种积的性质称为\emph{函子性}。你可以想象它允许你在积内部转换两个对象以获得新的积。我们还说函子性允许我们\index{lifting}\emph{提升}一对箭头以操作积。

 \section{Duality\\对偶性}

 当一个孩子看到一个箭头时,它知道哪一端指向源,哪一端指向目标
 \[a \to b \]
 但这也许只是一个先入之见。如果我们把$b$称为源,$a$称为目标,宇宙会有很大不同吗?

 我们仍然能够将这个箭头与这个箭头组合
 \[b \to c\]
 其“目标”$b$与$a \to b$的“源”相同,结果仍然是一个箭头
 \[a \to c\]
 只不过现在我们会说它从$c$到$a$。

 在这个对偶宇宙中,我们称之为“初始”的对象将被称为“终端”,因为它是从所有对象来的唯一箭头的“目标”。相反,终端对象将被称为初始对象。

 现在考虑我们用来定义和对象的这个图:
 \[
  \begin{tikzcd}
   a
   \arrow[dr,  bend left, "\text{Left}"']
   \arrow[ddr, bend right, "f"']
   && b
   \arrow[dl, bend right, "\text{Right}"]
   \arrow[ddl, bend left, "g"]
   \\
   &a + b
   \arrow[d, dashed, "h"]
   \\
   & c
  \end{tikzcd}
 \]
 在新的解释中,箭头$h$将从任意对象$c$“到”我们称为$a + b$的对象。这个箭头由一对箭头$(f, g)$唯一定义,其“源”是$c$。如果我们将$\text{Left}$重命名为$\text{fst}$,将$\text{Right}$重命名为$\text{snd}$,我们将得到积的定义图。

 积是和类型的对偶。

 相反,和类型是积的对偶。

 \medskip

 范畴论中的每个构造都有其对偶。

 \medskip

 如果箭头的方向只是解释问题,那么在编程中,和类型和积类型有什么区别呢?这种区别可以追溯到我们一开始做的一个假设:没有指向初始对象的箭头(除身份箭头外)。这与终端对象有大量指向外部的箭头形成对比,我们用这些箭头来定义(全局)元素。事实上,我们假设每个感兴趣的对象都有元素,而没有元素的对象是与\hask{Void}同构的。

 当我们讨论函数类型时,我们将看到更深的差异。

 \section{Monoidal Category\\幺半范畴}

 我们已经看到积满足以下简单规则:
 \begin{align*}
  1 \times a &\cong a
  \\
  a \times b &\cong b \times a
  \\
  (a \times b) \times c &\cong a \times (b \times c)
 \end{align*}
 并且是函子性的。

 定义了具有这些属性的运算的范畴称为\emph{对称幺半范畴}\footnote{严格来说,两个对象的积是同构定义的,而幺半范畴中的积必须是确切定义的。但通过选择一个积,我们可以得到一个幺半范畴}。当我们处理和类型和初始对象时,我们看到了类似的结构。

 一个范畴可以同时有多个幺半结构。当你不想为你的幺半结构命名时,你可以用张量符号代替加号或积号,用字母$I$代替中性元素。对称幺半范畴的规则可以写作:
 \begin{align*}
  I \otimes a &\cong a
  \\
  a \otimes b &\cong b \otimes a
  \\
  (a \otimes b) \otimes c &\cong a \otimes (b \otimes c)
 \end{align*}

 这些同构通常写作称为\index{associator}结合子的可逆箭头家族和\index{unitor}单位子。如果幺半范畴不是对称的,则有单独的左单位和右单位子。
 \begin{align*}
  \alpha &\colon (a \otimes b) \otimes c \to a \otimes (b \otimes c)
  \\
  \lambda &\colon I \otimes a \to a
  \\
  \rho &\colon a \otimes I \to a
 \end{align*}
 对称性由下式见证:
 \[ \gamma \colon a \otimes b \to b \otimes a \]
 函子性让我们\emph{提升}一对箭头:
 \begin{align*}
  f &\colon a \to a' \\
  g &\colon b \to b'
 \end{align*}
 以操作张量积:
 \[ a \otimes b \xrightarrow{f \otimes g} a' \otimes b' \]

 如果我们将态射视为动作,那么它们的张量积对应于同时执行两个动作。与态射的序列组合相比,这表明它们的时间顺序。

 你可以将张量积想象为积和和的最小公分母。它仍然有一个引入规则,需要两个对象$a$和$b$;但它没有消去规则。一旦创建,张量积就“忘记”了它是如何创建的。与笛卡尔积不同,它没有投影。

 一些有趣的张量积甚至不是对称的。

 \subsection{Monoids\\幺半群}

 幺半群是非常简单的结构,具有二元运算和一个单位。以加法和零为运算的自然数构成一个幺半群。以乘法和一为运算的自然数也构成一个幺半群。

 直觉上,幺半群让你将两件事物结合起来,得到另一件事物。还有一个特殊的事物,将其与任何其他事物结合,都会返回相同的事物。那就是单位元。并且结合必须是结合的。

 未被假定的是结合是对称的,或存在逆元素。

 定义幺半群的规则让人联想到范畴的规则。不同之处在于,在幺半群中,任何两件事物都是可组合的,而在范畴中通常不是这样:只有当一个箭头的目标是另一个箭头的源时,你才能组合这两个箭头。除非该范畴仅包含一个对象,在这种情况下,所有箭头都是可组合的。

 包含单个对象的范畴称为幺半群。组合运算是箭头的组合,单位是身份箭头。

 这是一个完全有效的定义。然而,在实践中,我们通常对嵌入更大范畴中的幺半群更感兴趣。特别是在编程中,我们希望能够在类型和函数的范畴内定义幺半群。

 但是,在范畴中,我们倾向于整体定义运算,而不是关注个别元素。因此我们从一个对象$m$开始。二元运算是一个有两个参数的函数。由于积的元素是成对的元素,我们可以将二元运算表征为从积$m \times m$到$m$的箭头:
 \[ \mu \colon m \times m \to m \]
 单位元素可以定义为从终端对象$1$的箭头:
 \[ \eta \colon 1 \to m \]

 我们可以直接将这种描述翻译为Haskell,定义一个\index{type class}类型类,它配备了两个方法,传统上称为\hask{mappend}和\hask{mempty}:
 \begin{haskell}
  class Monoid m where
  mappend :: (m, m) -> m
  mempty  :: () -> m
 \end{haskell}

 这两个箭头$\mu$和$\eta$必须满足幺半群律,但我们再次需要在整体上表述它们,而不涉及元素。

 为了表述左单位律,我们首先创建积$1 \times m$。然后我们使用$\eta$“选择$m$中的单位元素”,或者,用箭头的术语来说,将$1$变成$m$。由于我们正在操作积$1 \times m$,我们必须提升对$\langle \eta, id_m \rangle$,确保我们“不接触”$m$。最后,我们使用$\mu$执行“乘法”。

 我们希望结果与原始的$m$元素相同,但不提及元素。因此,我们只是使用左单位元$\lambda$从$1 \times m$到$m$而不“搅动事物。”
 \[
  \begin{tikzcd}
   1 \times m
   \arrow[rr, "\eta \times id_m"]
   \arrow[rrd, "\lambda"']
   & & m \times m
   \arrow[d, "\mu"]
   \\
   && m
  \end{tikzcd}
 \]
 这里是右单位的类似律:
 \[
  \begin{tikzcd}
   m \times m
   \arrow[d, "\mu"]
   && m \times 1
   \arrow[ll, "id_m \times \eta"']
   \arrow[lld, "\rho"]
   \\
   m
  \end{tikzcd}
 \]
 为了表述结合律,我们必须从三重积开始,并在整体上操作。在这里,$\alpha$是结合子,它重新排列积而不“搅动事物”。
 \[
  \begin{tikzcd}
  (m \times m) \times m
  \arrow[rr, "\alpha"]
  \arrow[d, "\mu \times id"]
  &&
  m \times (m \times m)
  \arrow[d, "id \times \mu"]
  \\
  m \times m
  \arrow[dr, "\mu"]
  & & m \times m
  \arrow[dl, "\mu"']
  \\
  &  m
  \end{tikzcd}
 \]

 注意,我们不需要对我们与对象$m$和$1$一起使用的范畴积假设太多,特别是我们从未需要使用投影。这表明上述定义同样适用于任意幺半范畴中的张量积。它甚至不需要是对称的。我们只需假设:存在一个单位对象,积是函子的,并且它满足单位和结合律(最多同构)。

 因此,如果我们将$\times$替换为$\otimes$,并将$1$替换为$I$,我们将得到任意幺半范畴中幺半群的定义。

 任意幺半范畴中的\index{monoid}\emph{幺半群}是一个配备两个态射的对象$m$:
 \[ \mu \colon m \otimes m \to m \]
 \[ \eta \colon I \to m \]
 满足单位和结合律:
 \[
  \begin{tikzcd}
   1 \otimes m
   \arrow[rr, "\eta \otimes id_m"]
   \arrow[rrd, "\lambda"']
   & & m \otimes m
   \arrow[d, "\mu"']
   && m \otimes 1
   \arrow[ll, "id_m \otimes \eta"']
   \arrow[lld, "\rho"]
   \\
   && m
  \end{tikzcd}
 \]

 \[
  \begin{tikzcd}
  (m \otimes m) \otimes m
  \arrow[rr, "\alpha"]
  \arrow[d, "\mu \otimes id_m"]
  &&
  m \otimes (m \otimes m)
  \arrow[d, "id_m \otimes \mu"]
  \\
  m \otimes m
  \arrow[dr, "\mu"]
  & & m \otimes m
  \arrow[dl, "\mu"']
  \\
  &  m
  \end{tikzcd}
 \]
 我们使用了$\otimes$的函子性来提升箭头对,例如$\eta \otimes id_m$、$\mu \otimes id_m$等。

\end{document}
