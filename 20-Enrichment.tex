\documentclass[DaoFP]{subfiles}
\begin{document}
 \setcounter{chapter}{19}

 \chapter{Enrichment(丰盈)}

 老子曰:“知足者富。”

 \section{Enriched Categories(丰盈范畴)}

 这可能会让人感到惊讶,但在没有丰盈范畴的背景知识下,是无法完全解释\hask{Functor}(函子)的Haskell定义的。在本章中,我将尝试展示,从概念上来说,丰盈并不是从普通范畴论向前迈出的一大步。

 研究丰盈范畴的额外动机来自这样一个事实:大量文献,尤其是网站nLab,通常以最一般的术语(这通常意味着以丰盈范畴的术语)来描述概念。大多数常见的构造都可以通过更改词汇来翻译,将同态集(hom-sets)替换为同态对象(hom-objects),将$\Set$替换为一个幺半范畴$\cat V$。

 一些丰盈概念,如加权极限和余极限(weighted limits and colimits),本身就非常强大,以至于人们可能会被引诱将Mac Lane的格言“所有概念都是Kan扩张”替换为“所有概念都是加权(余)极限。”

 \subsection{Set-theoretical foundations(集合论基础)}

 范畴论在其基础上非常节俭。但它(不情愿地)依赖于集合论。特别是,同态集的概念被定义为两个对象之间箭头的集合,这引入了集合论作为范畴论的前提。诚然,箭头只在\emph{局部小}范畴中形成集合,但这只是聊以自慰,考虑到处理太大的东西而无法成为集合需要更多的理论。

 如果范畴论能够自举,例如通过将同态集替换为更一般的对象,那就太好了。这正是丰盈范畴背后的思想。然而,这些同态对象必须来自某个具有同态集的其他范畴,并且在某些时候我们必须回到集合论基础上。然而,能够用不同的东西替换没有结构的同态集,扩大了我们建模更复杂系统的能力。

 集合的主要属性是,与对象不同,它们不是原子的:它们有\emph{元素}。在范畴论中,我们有时谈论\emph{广义元素},它们只是指向某个对象的箭头;或者\emph{全局元素},它们是来自终对象(有时来自幺元对象$I$)的箭头。但最重要的是,集合定义了元素的\emph{相等性}。

 我们几乎学到的所有关于范畴的知识都可以翻译到丰盈范畴的领域中。然而,许多范畴推理涉及交换图表,它们表示箭头的相等性。在丰盈环境中,我们没有对象之间的箭头,因此所有这些构造都必须进行修改。

 \subsection{Hom-Objects(同态对象)}

 乍一看,用对象替换同态集似乎是一个退步。毕竟,集合有元素,而对象只是无形的斑点。然而,同态对象的丰富性体现在它们所属范畴的态射中。从概念上讲,集合是无结构的,意味着它们之间有大量的态射(函数)。态射越少,通常意味着结构越多。

 定义丰盈范畴的指导原则是,我们应该能够将普通范畴论作为特例来恢复。毕竟,同态集\emph{是}$\Set$范畴中的对象。实际上,我们非常努力地将集合的属性表达为函数而不是元素。

 话虽如此,用复合和单位定义范畴时会涉及到同态集的\emph{元素}的态射。因此,让我们首先在不诉诸元素的情况下重新表述范畴的基本原理。

 箭头的复合可以作为同态集之间的函数来定义:
 \[ \circ \colon \mathcal C (b, c) \times \mathcal C (a, b) \to \mathcal C (a, c) \]

 我们可以使用单集合中的函数来代替单位箭头:
 \[ j_a \colon 1 \to \mathcal C (a, a) \]

 这表明,如果我们想用来自某个范畴$\mathcal V$的对象替换同态集$\mathcal C (a, b)$,我们必须能够\emph{乘}这些对象以定义复合,并且我们需要某种\emph{单位对象}来定义单位。我们可以要求$\mathcal V$是笛卡尔的,但实际上,幺半范畴就足够了。正如我们将看到的,幺半范畴的单位和结合律直接转换为复合的单位和结合律。

 \subsection{Enriched Categories(丰盈范畴)}

 设$\mathcal V$是一个具有张量积$\otimes$、单位对象$I$和结合子及两个单位子的幺半范畴(以及它们的逆):
 \begin{align*}
  \alpha &\colon (a \otimes b) \otimes c \to a \otimes (b \otimes c)
  \\
  \lambda &\colon I \otimes a \to a
  \\
  \rho &\colon a \otimes I \to a
 \end{align*}
 一个丰盈于$\mathcal V$的范畴$\mathcal C$具有对象,并且对于任意一对对象$a$和$b$,有一个同态对象$\mathcal C(a, b)$。这个同态对象是$\mathcal V$中的一个对象。复合使用$\mathcal V$中的箭头定义:
 \[ \circ \colon \mathcal C (b, c) \otimes \mathcal C (a, b) \to \mathcal C (a, c) \]
 单位由箭头定义:
 \[ j_a \colon I \to \mathcal C (a, a) \]
 结合律由$\mathcal V$中的结合子表达:
 \[
  \begin{tikzcd}
   \big( \mathcal C (c, d) \otimes \mathcal C (b, c) \big) \otimes \mathcal C(a, b)
   \arrow[rr, "\alpha"]
   \arrow[d, "\circ \otimes id"]
   &&  \mathcal C (c, d) \otimes \big( \mathcal C (b, c) \otimes \mathcal C(a, b) \big)
   \arrow[d, "id \otimes \circ"]
   \\
   \mathcal C(b, d) \otimes \mathcal C(a, b)
   \arrow[dr, "\circ"]
   && \mathcal C(c, d) \otimes \mathcal C(a, c)
   \arrow[dl, "\circ"']
   \\
   & \mathcal C(a, d)
  \end{tikzcd}
 \]
 单位律则由$\mathcal V$中的单位子表达:

 \[
  \begin{tikzcd}
   I \otimes \mathcal C(a, b)
   \arrow[r, "\lambda"]
   \arrow[d, "j_b \otimes id"']
   & \mathcal C(a, b)
   \\
   \mathcal C(b, b) \otimes \mathcal C(a, b)
   \arrow[ru, "\circ"]
  \end{tikzcd}
  \qquad
  \begin{tikzcd}
   \mathcal C(a, b) \otimes I
   \arrow[r, "\rho"]
   \arrow[d, "id \otimes j_a"']
   & \mathcal C(a, b)
   \\
   \mathcal C(a, b) \otimes \mathcal C(a, a)
   \arrow[ru, "\circ"]
  \end{tikzcd}
 \]
 注意,这些都是$\mathcal V$中的图表,在这里我们确实有形成同态集的箭头。我们仍然在不同的层次上回到了集合论。

 一个丰盈于$\mathcal V$的范畴也称为$\mathcal V$-范畴。接下来我们将假设丰盈范畴是\emph{对称}幺半范畴,因此我们可以形成相反范畴和积$\mathcal V$-范畴。

 与$\mathcal C$对立的$\mathcal V$-范畴$\mathcal C^{op}$通过逆转同态对象获得,即:
 \[ \mathcal C^{op}(a, b) = \mathcal C(b, a) \]
 对立范畴中的复合涉及逆转同态对象的顺序,因此它只有在张量积是对称的情况下才有效。

 我们还可以定义$\mathcal V$-范畴的张量积;同样,前提是$\mathcal V$是对称的。两个$\mathcal V$-范畴$\mathcal C \otimes \mathcal D$的积具有对象对,分别来自每个范畴。这些对之间的同态对象被定义为张量积:
 \[ \mathcal (C \otimes \mathcal D) (\langle c, d \rangle, \langle c', d' \rangle) = \mathcal C(c, c') \otimes \mathcal D (d, d') \]
 我们需要张量积的对称性来定义复合。事实上,我们需要在中间交换两个同态对象,然后才能应用两个可用的复合:
 \[ \circ \colon  \big(\mathcal C(c', c'',) \otimes \mathcal D (d', d'')\big) \otimes \big( \mathcal C(c, c') \otimes \mathcal D (d, d')\big) \to  \mathcal C(c, c'') \otimes \mathcal D (d, d'') \]
 单位箭头是两个单位的张量积:
 \[ I_{\mathcal C} \otimes I_{\mathcal D} \xrightarrow{j_c \otimes j_d} \mathcal C(c, c) \otimes \mathcal D (d, d) \]


 \begin{exercise}
  在$\mathcal V$-范畴$\mathcal C^{op}$中定义复合和单位。
 \end{exercise}

 \begin{exercise}
  证明每个$\mathcal V$-范畴$\mathcal C$都有一个底层的普通范畴$\mathcal C_0$,其对象相同,但其同态集由同态对象的(幺半全局)元素给出,即$\mathcal V(I, \mathcal C(a, b))$的元素。
 \end{exercise}

 \subsection{Examples(示例)}

 从这个新的视角来看,我们迄今为止研究的普通范畴可以说是通过单子积$\times$和单位集$1$在幺半范畴$(\Set, \times, 1)$中得到了丰盈的。

 有趣的是,2-范畴可以被视为在$\mathbf{Cat}$上丰盈。实际上,2-范畴中的1-态射本身就是另一个范畴中的对象。2-态射就是该范畴中的箭头。特别地,小范畴的2-范畴$\mathbf{Cat}$在自身中得到了丰盈。其同态对象是函子范畴,这些对象位于$\mathbf{Cat}$中。

 \subsection{Preorders(预序)}

 丰盈并不总是意味着增加更多的内容。有时它看起来更像是一种贫乏,比如在一个行箭范畴上的丰盈情况。

 这个范畴只有两个对象,为了这个构造的目的,我们称它们为$\text{False}$和$\text{True}$。从$\text{False}$到$\text{True}$有一个箭头(不计算单位箭头),这使得$\text{False}$成为初始对象,$\text{True}$成为终对象。
 \[
  \begin{tikzcd}
   \text{False}
   \arrow[r, "!"]
   \arrow[loop, "id_{\text{False}}"']
   & \text{True}
   \arrow[loop, "id_{\text{True}}"']
  \end{tikzcd}
 \]
 要将其变为幺半范畴,我们定义张量积,使得:
 \[ \text{True} \otimes \text{True} = \text{True} \]
 其他所有组合都产生\text{False}。
 $\text{True}$是幺元,因为:
 \[ \text{True} \otimes x = x \]

 在幺半行箭范畴上丰盈的范畴称为\emph{预序}。任何两个对象之间的同态对象$\mathcal C (a, b)$可以是$\text{False}$或$\text{True}$。我们解释$\text{True}$为$a$在预序中领先于$b$,即$a \le b$。$\text{False}$表示这两个对象不相关。

 复合的一个重要属性是,如果左边的两个同态对象都是$\text{True}$,那么右边也必须是$\text{True}$。(它不能是$\text{False}$,因为没有箭头从$\text{True}$到$\text{False}$。)在预序解释中,这意味着$\le$是传递的:
 \[ b \le c \land a \le b \implies a \le c \]

 通过相同的推理,存在单位箭头:
 \[ j_a \colon \text{True} \to \mathcal C(a, a) \]
 意味着$\mathcal C(a, a)$总是$\text{True}$。在预序解释中,这意味着$\le$是自反的,即$a \le a$。

 注意,预序不排除循环,特别是可以有$a \le b$且$b \le a$但$a$不等于$b$。

 也可以在不诉诸丰盈的情况下定义预序,作为\index{thin category}\emph{薄范畴}——在其中,任何两个对象之间最多有一个箭头。

 \subsection{Self-enrichment(自丰盈)}

 任何笛卡尔闭范畴$\mathcal V$都可以视为自丰盈的。这是因为每个外部同态集$\mathcal C(a, b)$都可以被内部同态$b^a$(箭头的对象)替换。

 实际上,每个\emph{幺半闭}范畴$\mathcal V$都是自丰盈的。回想一下,在幺半闭范畴中我们有同态函子伴随关系:
 \[ \mathcal V (a \otimes b, c) \cong \mathcal V (a, [b, c]) \]
 此伴随的余单元作为评估态射起作用:
 \[ \varepsilon_{b c} \colon [b, c] \otimes b \to c \]

 要在这个自丰盈的范畴中定义复合,我们需要一个箭头:
 \[ \circ \colon [b, c] \otimes [a, b] \to [a, c] \]
 诀窍是考虑整个同态集,并表明我们始终可以在其中选取一个规范元素。我们从集合开始:
 \[ \mathcal V([b, c] \otimes [a, b], [a, c]) \]
 我们可以使用伴随关系将其重写为:
 \[  \mathcal V( ([b, c] \otimes [a, b]) \otimes a, c) \]
 现在我们只需从这个同态集中选取一个元素即可。我们通过构造以下组合来完成:
 \[ ([b, c] \otimes [a, b]) \otimes a \xrightarrow{\alpha}
 [b, c] \otimes ([a, b] \otimes a) \xrightarrow{id \otimes \varepsilon_{a b} }
 [b, c] \otimes b \xrightarrow{\varepsilon_{b c}} c \]
 我们使用了结合子和伴随关系的余单元。

 我们还需要一个箭头来定义单位:
 \[ j_a \colon I \to [a, a] \]
 同样,我们可以将其作为同态集$\mathcal V(I, [a, a])$的成员选取。我们使用伴随关系:
 \[ \mathcal V(I, [a, a]) \cong \mathcal V (I \otimes a, a) \]
 我们知道这个同态集包含左幺子$\lambda$,因此我们可以用它来定义$j_a$。

 \section{$\mathcal V$-Functors($\mathcal V$-函子)}

 普通函子将对象映射到对象,将箭头映射到箭头。类似地,一个丰盈函子$F$将对象映射到对象,但它必须将同态对象映射到同态对象。只有当源范畴$\mathcal C$中的同态对象属于与目标范畴$\mathcal D$中的同态对象相同的范畴时,这才有可能。换句话说,这两个范畴必须丰盈于相同的$\mathcal V$。$F$在同态对象上的作用使用$\mathcal V$中的箭头定义:
 \[ F_{a b} \colon \mathcal C (a, b) \to \cat D (F a, F b) \]
 为清晰起见,我们在$F$的下标中指定了对象对。

 函子必须保持复合和单位。可以将它们表示为$\mathcal V$中的交换图表:

 \[
  \begin{tikzcd}
   \mathcal C(b, c) \otimes \mathcal C(a, b)
   \arrow[r, "\circ"]
   \arrow[d, "F_{b c} \otimes F_{a b}"]
   & \mathcal C(a, c)
   \arrow[d, "F_{a c}"]
   \\
   \mathcal D(F b, F c) \otimes \mathcal D (F a, F b)
   \arrow[r, "\circ"]
   & \mathcal D(F a, F b)
  \end{tikzcd}
  \qquad
  \begin{tikzcd}
   & I
   \arrow[dl, "j_a"']
   \arrow[dr, "j_{F a}"]
   \\
   \mathcal C(a, a)
   \arrow[rr, "F_{a a}"]
   && \mathcal D( F a, F a)
  \end{tikzcd}
 \]
 注意我对两个不同的复合使用了相同的符号$\circ$,对两个不同的单位映射使用了相同的$j$。它们的含义可以从上下文中推导出来。

 和以前一样,所有的图表都在范畴$\mathcal V$中。

 \subsection{The \text{Hom}-functor(同态函子)}

 在一个丰盈于幺半\emph{闭}范畴$\mathcal V$的范畴中,同态函子是一个丰盈函子:
 \[ \text{Hom}_{\cat C} \colon \mathcal C^{op} \otimes \mathcal C \to \mathcal V \]
 在这里,为了定义一个丰盈函子,我们必须将$\mathcal V$视为自丰盈。

 很明显,这个函子在(对象对)上的作用是如何进行的:
 \[ \text{Hom}_{\cat C} \langle a, b \rangle = \mathcal C (a, b) \]


 为了定义一个丰盈函子,我们必须定义同态函子对同态对象的作用。在这里,源范畴是$\mathcal C^{op} \otimes \mathcal C$,目标范畴是$\mathcal V$,两者都丰盈于$\mathcal V$。让我们考虑从$\langle a, a' \rangle$到$\langle b, b' \rangle$的同态对象。同态函子对该同态对象的作用是$\mathcal V$中的一个箭头:
 \[ \text{Hom}_{\langle a, a' \rangle \langle b, b' \rangle} \colon (C^{op} \otimes \mathcal C)(\langle a, a' \rangle, \langle b, b' \rangle) \to \mathcal V (\text{Hom}\langle a, a' \rangle, \text{Hom}\langle b, b' \rangle)\]
 根据积范畴的定义,源是两个同态对象的张量积。目标是$\mathcal V$中的内部同态。我们正在寻找的箭头是:
 \[ \mathcal C(b, a) \otimes \mathcal C(a', b') \to [\mathcal C(a, a'), \mathcal C(b, b')] \]
 我们可以使用柯里化的同态函子伴随关系来展开内部同态:
 \[ \Big( \mathcal C(b, a) \otimes \mathcal C(a', b') \Big) \otimes \mathcal C(a, a') \to \mathcal C(b, b') \]
 我们可以通过重新排列积并两次应用复合来构造这个箭头。

 在丰盈环境中,定义从$a$到$b$的单个态射最接近的方法是使用一个来自单位对象的箭头。我们将一个同态对象的(幺半全局)元素定义为$\mathcal V$中的一个态射:
 \[ f \colon I \to \mathcal C(a, b) \]
 我们可以定义提升此箭头的含义,使用同态函子。例如,将第一个参数保持为常量,我们可以定义:
 \[ \mathcal C(c, f) \colon \mathcal C(c, a) \to C(c, b) \]
 作为以下组合:
 \[ \mathcal C(c, a) \xrightarrow{\lambda^{-1}} I \otimes \mathcal C(c, a) \xrightarrow{f \otimes id} \mathcal C(a, b) \otimes \mathcal C(c, a) \xrightarrow{\circ} \mathcal C(c, b) \]
 类似地,$f$的反变提升:
 \[ \mathcal C(f, c) \colon \mathcal C(b, c) \to \mathcal C(a, c) \]
 可以定义为:
 \[ \mathcal C(b, c) \xrightarrow{\rho^{-1}} \mathcal C(b, c) \otimes I \xrightarrow{id \otimes f} \mathcal C (b, c) \otimes \mathcal C(a, b) \xrightarrow{\circ} \mathcal C(a, c) \]

 我们在普通范畴论中学习的许多熟悉构造都有其丰盈对应物,产品被张量积取代,$\Set$被$\mathcal V$取代。

 \begin{exercise}
  什么是两个预序之间的函子?
 \end{exercise}

 \subsection{Enriched co-presheaves(丰盈余预层)}

 余预层,即$\Set$-值函子,在范畴论中扮演着重要角色,因此很自然地问在丰盈环境中的对应物是什么。余预层的推广是$\mathcal V$-函子$\mathcal C \to \mathcal V$。只有当$\mathcal V$可以成为一个$\mathcal V$-范畴,即当它是幺半闭的时,这才有可能。

 一个丰盈余预层将$\mathcal C$的对象映射到$\mathcal V$的对象,并将$\mathcal C$的同态对象映射到$\mathcal V$的内部同态:
 \[ F_{a b} \colon \mathcal C (a, b) \to \cat [F a, F b] \]

 特别地,同态函子是一个$\mathcal V$-值$\mathcal V$-函子的例子:
 \[ \text{Hom} \colon \cat C^{op} \otimes \cat C \to \cat V \]

 同态函子是丰盈态射的特例,它被定义为:
 \[ \mathcal C^{op} \otimes \mathcal D \to \mathcal V \]

 \begin{exercise}
  张量积是$\mathcal V$中的一个函子:
  \[ \otimes \colon \cat V \times \cat V \to \cat V \]
  证明如果$\mathcal V$是幺半闭的,张量积定义了一个$\mathcal V$-函子。提示:定义它对内部同态的作用。
 \end{exercise}

 \subsection{Functorial strength and enrichment(函子强度与丰盈)}

 当我们讨论monads(幺半群)时,我提到一个使它们在编程中起作用的重要属性。定义monads的端函子必须是强大的,这样我们才能在monadic代码中访问外部上下文。

 事实证明,我们在Haskell中定义端函子的方式使它们自动变得强大。原因是强度与丰盈相关,正如我们所见,笛卡尔闭范畴是自丰盈的。让我们从一些定义开始。

 在幺半范畴中,端函子的函子强度定义为具有如下分量的自然变换:
 \[ \sigma_{a b} \colon a \otimes F(b) \to F (a \otimes b) \]
 有一些显而易见的一致性条件使得强度尊重张量积的性质。这是结合性条件:
 \[
  \begin{tikzcd}
  (a \otimes b) \otimes F (c)
  \arrow[rr, "\sigma_{(a \otimes b) c}"]
  \arrow[d, "\alpha"]
  && F((a \otimes b) \otimes c)
  \arrow[d, "F(\alpha)"]
  \\
  a \otimes (b \otimes F(c))
  \arrow[r, "a \otimes \sigma_{b c}"]
  & a \otimes F(b \otimes c)
  \arrow[r, "\sigma_{a (b \otimes c)}"]
  & F(a \otimes (b \ot织c))
  \\
  \end{tikzcd}
 \]
 这是单位条件:
 \[
  \begin{tikzcd}
   I \otimes F(a)
   \arrow[r, "\sigma_{I a}"]
   \arrow[rd, "\lambda"]
   & F(I \otimes a)
   \arrow[d, "F (\lambda)"]
   \\
   & F(a)
  \end{tikzcd}
 \]

 在一般幺半范畴中,这被称为\emph{左强度},并且有一个对应的右强度定义。在对称幺半范畴中,这两者是等价的。

 一个丰盈端函子将同态对象映射到同态对象:
 \[ F_{a b} \colon \cat C (a, b) \to \cat C (F a, F b) \]
 如果我们将幺半闭范畴$\mathcal V$视为自丰盈,内部同态就是内部同态,因此丰盈端函子配备了以下映射:
 \[ F_{a b} \colon [a, b] \to [F a, F b] \]
 将其与Haskell中的\hask{Functor}定义进行比较:
 \begin{haskell}
  class Functor f where
  fmap :: (a -> b) -> (f a -> f b)
 \end{haskell}
 此定义中涉及的函数类型,\hask{(a -> b)}和\hask{(f a -> f b)},是\emph{内部}同态。因此,Haskell的\hask{Functor}确实是一个丰盈函子。

 在Haskell中,我们通常不区分外部和内部同态,因为它们的元素集是同构的。这是柯里化伴随关系的一个简单结果:
 \[ \cat C(1 \times b, c) \cong \cat C(1, [b, c]) \]
 以及终对象是笛卡尔积的单位这一事实。

 事实证明,在自丰盈范畴$\mathcal V$中,每个强端函子都是自动丰盈的。实际上,为了证明一个函子是丰盈的,我们需要定义内部同态之间的映射,即一个同态集的元素:
 \[ F_{a b} \in \cat V([a, b], [F a, F b]) \]
 使用同态伴随,这等同于:
 \[ \cat V([a, b] \otimes F a, F b) \]
 我们可以通过将强度和伴随的余单元(评估态射)组合起来构造此映射:
 \[ [a, b] \otimes F a \xrightarrow{\sigma_{[a, b] a}} F ([a, b] \otimes a) \xrightarrow{F \epsilon_{a b}} F b \]

 反过来,每个$\mathcal V$中的丰盈端函子都是强的。为了展示强度,我们需要定义映射$\sigma_{a b}$,或等价地(通过同态伴随):
 \[ a \to [F b, F (a \otimes b)] \]
 回想一下同态伴随关系的单位定义,协同评估态射:
 \[ \eta_{a b} \colon a \to [b, a \otimes b] \]
 我们构造以下复合:
 \[ a \xrightarrow{\eta_{a b}} [b, a \otimes b] \xrightarrow{F_{b, a \otimes b}} [F b, F (a \otimes b)] \]

 这可以直接翻译成Haskell:
 \begin{haskell}
  strength :: Functor f => (a, f b) -> f (a, b)
  strength = uncurry (\a -> fmap (coeval a))
 \end{haskell}
 协同评估的定义如下:
 \begin{haskell}
  coeval :: a -> (b -> (a, b))
  coeval a = \b -> (a, b)
 \end{haskell}
 由于柯里化和评估内置在Haskell中,我们可以进一步简化这个公式:
 \begin{haskell}
  strength :: Functor f => (a, f b) -> f (a, b)
  strength (a, bs) = fmap (a, ) bs
 \end{haskell}

 \section{$\mathcal V$-Natural Transformations($\mathcal V$-自然变换)}

 两个从$\mathcal C$到$\mathcal D$的函子$F$和$G$之间的普通自然变换是从同态集中选择箭头$\mathcal D(F a, G a)$。在丰盈环境中,我们没有箭头,因此我们能做的最好的事情就是使用单位对象$I$来进行选择。我们在$a$处定义$\mathcal V$-自然变换的一个分量作为一个箭头:
 \[ \nu_a \colon I \to \mathcal D(F a, G a) \]

 自然性条件有点棘手。标准的自然性方块涉及到一个任意箭头$f \colon a \to b$的提升,以及以下组合的相等性:
 \[ \nu_b \circ F f = G f \circ \nu_a \]

 让我们考虑这个方程中涉及的同态集。我们正在提升一个态射$f \in \mathcal C(a, b)$。方程两侧的组合都是$\mathcal D(F a, G b)$的元素。

 在左侧,我们有箭头$\nu_b \circ F f$。复合本身是从两个同态集的积映射的:
 \[  \mathcal D(F b, G b) \times \mathcal D(F a, F b) \to \mathcal D(F a, G b) \]
 同样地,在右侧我们有$G f \circ \nu_a$,它是一个复合:
 \[ \mathcal D(G a, G b) \times \mathcal D(F a, G a) \to  \mathcal D(F a, G b) \]

 在丰盈环境中,我们必须处理同态对象而不是同态集,并且自然变换的分量是使用单位$I$选择的。我们始终可以使用左或右幺子的逆凭空生成单位。

 总而言之,自然性条件表示为以下交换图表:
 \[
  \begin{tikzcd}
   & I \otimes \mathcal C(a, b)
   \arrow[r, "\nu_b \otimes F_{a b}"]
   & \mathcal D(F b, G b) \otimes \mathcal D(F a, F b)
   \arrow[dr, "\circ"]
   \\
   \mathcal C(a, b)
   \arrow[ur, "\lambda^{-1}"]
   \arrow[dr, "\rho^{-1}"']
   &&& \mathcal D(F a, G b)
   \\
   & \mathcal C(a, b) \otimes I
   \arrow[r, "G_{a b} \otimes \nu_a"]
   & \mathcal D(G a, G b) \otimes \mathcal D(F a, G a)
   \arrow[ur, "\circ"']
  \end{tikzcd}
 \]
 这也适用于普通范畴,在这里我们可以通过首先从$\mathcal C(a, b)$中选择一个$f$来跟踪此图表中的两条路径。然后我们可以使用$\nu_b$和$\nu_a$选择自然变换的分量。我们还可以使用$F$或$G$提升$f$。最后,我们使用复合来重现自然性方程。

 如果我们使用早先定义的同态函子对同态对象的全局元素的作用,此图表可以进一步简化。自然变换的分量定义为这样的全局元素:
 \[ \nu_a \colon I \to \mathcal D(F a, G a) \]
 我们可以使用两种这样的提升:
 \[ \mathcal D(d, \nu_b) \colon \mathcal D(d, F b) \to \mathcal D(d, G b) \]
 和:
 \[ \mathcal D (\nu_a, d) \colon \mathcal D(G a, d) \to \mathcal D(F a, d) \]
 我们得到的东西更像是熟悉的自然性方块:
 \[
  \begin{tikzcd}
   & \mathcal D(F a, F b)
   \arrow[dr, "{\mathcal D (F a, \nu_b)}"]
   \\
   \mathcal C(a, b)
   \arrow[ur, "F_{a b}"]
   \arrow[dr, "G_{a b}"']
   && \mathcal D(F a, G b)
   \\
   & \mathcal D (G a, G b)
   \arrow[ur, "{\mathcal D (\nu_a, G b)}"']
  \end{tikzcd}
 \]


 $F$和$G$之间的$\mathcal V$-自然变换形成了一个我们称之为$\mathcal V\text{-nat} (F, G)$的集合。

 我们之前看到,在普通范畴中,自然变换的集合可以写成一个端:
 \[ [\mathcal C, \mathcal D](F, G) \cong \int_a \mathcal D(F a, G a) \]
 事实证明,端和余端可以为丰盈态射定义,因此该公式适用于丰盈自然变换。不同之处在于,它不是定义自然变换的\emph{集合}$\mathcal V\text{-nat} (F, G)$,而是定义了$\mathcal V$中的自然变换的\emph{对象} $[\mathcal C, \mathcal D](F, G)$。

 $\mathcal V$-态射$\mathcal C^{op} \otimes \mathcal C \to \mathcal V$的端的定义类似于我们在普通态射中看到的定义。例如,端是$\mathcal V$中的一个对象$e$,它配备了$\pi \colon e \to P$的超自然变换,在这些对象中是普遍的。

 \section{Yoneda Lemma(Yoneda引理)}

 普通的Yoneda引理涉及一个$\Set$-值函子$F$和自然变换的集合:
 \[ [\mathcal C, \Set]( \mathcal C(c, -), F) \cong F c \]
 为了将其推广到丰盈环境,我们将考虑一个$\mathcal V$-值函子$F$。和以前一样,只要它是闭的,我们就可以将$\mathcal V$视为自丰盈,因此我们可以谈论$\mathcal V$-值$\mathcal V$-函子。

 弱版本的Yoneda引理处理的是$\mathcal V$-自然变换的\emph{集合}。因此,我们必须将右侧转换为集合。这是通过获取$F c$的(幺半全局)元素来完成的。我们得到:
 \[ \mathcal V\text{-nat} ( \mathcal C(c, -), F) \cong \mathcal V(I, F c) \]

 强版本的Yoneda引理处理的是$\mathcal V$的对象,并使用$\mathcal V$中内部同态上的端来表示自然变换的对象:
 \[ \int_x [\mathcal C( c, x), F x] \cong F c \]

 \section{Weighted Limits(加权极限)}

 极限(和余极限)围绕着交换三角形构建,因此它们无法立即翻译到丰盈环境中。问题是,锥体是由“线”,即单个态射构成的。你可以将同态集视为厚线束,每条线的厚度为零。在构造锥体时,您选择同态集中的单根线。我们必须用更厚的东西替换这些线。

 考虑一个图表,即从索引范畴$\mathcal J$到目标范畴$\mathcal C$的函子。锥体的顶点$x$的线是从同态集$\mathcal C(x, D j)$中选择的,其中$j$是$\mathcal J$的一个对象。

 \[
  \begin{tikzcd}
   \\
   \\
   j
   \arrow[rr, red]
   \arrow[rd, red]
   && k
   \arrow[dl, red]
   \\
   & l
  \end{tikzcd}
  \qquad
  \begin{tikzcd}
   & x
   \arrow[ddl, ""']
   \arrow[ddl, bend right, "{\mathcal C(x, D j)}"']
   \arrow[ddl, bend left, ""]
   \\
   \\
   D j
   \arrow[rr, red]
   \arrow[rd, red]
   && D k
   \arrow[dl, red]
   \\
   & D l
  \end{tikzcd}
 \]
 这种选择的$j$'线可以描述为从单集合$1$的函数:
 \[ \gamma_j \colon 1 \to \mathcal C(x, D j) \]
 我们可以尝试将这些函数汇集成一个自然变换:
 \[ \gamma \colon \Delta_1 \to \mathcal C(x, D -) \]
 其中$\Delta_1$是一个常量函子,将$\mathcal J$的所有对象映射到单集合。自然性条件确保形成锥体侧面的三角形是交换的。

 顶点为$x$的所有锥体的集合由自然变换的集合给出:
 \[ [\mathcal J, \Set] (\Delta_1, \mathcal C(x, D-)) \]

 此重构使我们更接近丰盈环境,因为它用同态集而不是单个态射来重新表述问题。我们可以从考虑$\mathcal J$和$\mathcal C$丰盈于$\mathcal V$开始,在这种情况下$D$将是一个$\mathcal V$-函子。

 只有一个问题:我们如何定义一个常量$\mathcal V$-函子$\Delta_x \colon \cat C \to \cat D$?它对对象的作用是显而易见的:它将$\cat C$中的所有对象映射到$\cat D$中的一个对象$x$。但是,它对同态对象应该做什么?

 一个普通的常量函子$\Delta_x$将$\cat C(a, b)$中的所有态射映射到$\cat D(x, x)$中的单位。在丰盈环境中,然而,$\cat D(x, x)$是一个没有内部结构的对象。即使它碰巧是单位$I$,也无法保证对于每个同态对象$\mathcal C(a, b)$我们都可以找到一个箭头到$I$;即使有,也不一定是唯一的。换句话说,没有理由相信$I$是终对象。

 解决方案是“涂抹奇点”:我们应该使用其他的“加权”函子$W \colon \mathcal J\to \Set$来选择一个更厚的“圆柱”而不是使用常量函子选择单根线。顶点为$x$的加权锥体是自然变换集合的元素:
 \[ [\mathcal J, \Set] \left(W, \mathcal C(x, D-)\right) \]

 一个\emph{加权极限},也称为\index{indexed limit}\emph{索引极限}$\text{lim}^W D$,然后定义为普遍加权锥体。它意味着对于具有顶点$x$的任何加权锥体,存在唯一的从$x$到$\text{lim}^W D$的态射来对其进行因子分解。通过定义加权极限的同态集合的自然性来保证因子分解:
 \[  \mathcal C(x, \text{lim}^W D) \cong [\mathcal J, \Set] (W, \mathcal C(x, D-)) \]

 常规的非加权极限通常称为\index{limit, conical}\emph{锥形}极限,它对应于使用常量函子作为权重。

 这个定义可以几乎逐字翻译到丰盈环境中,通过将$\Set$替换为$\mathcal V$:
 \[  \mathcal C(x, \text{lim}^W D) \cong [\mathcal J, \mathcal V] (W, \mathcal C(x, D-)) \]
 当然,这个公式中的符号的含义已经改变。两侧现在都是$\mathcal V$中的对象。左侧是$\mathcal C$中的同态对象,右侧是两个$\mathcal V$-函子之间的自然变换对象。

 对偶地,定义一个加权余极限为自然同构:
 \[  \mathcal C(\text{colim}^W D, x) \cong [\mathcal J^{op}, \mathcal V] (W, \mathcal C(D-, x)) \]
 在这里,余极限由来自对立范畴的函子$W \colon \mathcal J^{op} \to \mathcal V$加权。

 加权(余)极限,在普通范畴和丰盈范畴中,都扮演着根本性角色:它们可以用于重新表述许多熟悉的构造,如(余)端,Kan扩张等。

 \section{Ends as Weighted Limits(端作为加权极限)}

 一个端与积或更一般的极限有很多共同点。如果你仔细看,投影$\pi_x \colon e \to P \langle a, a \rangle$形成了锥体的侧面;除了交换三角形,我们还有楔子。事实证明,我们可以将端表达为加权极限。这样表述的优点是它也适用于丰盈环境。

 我们已经看到,$\mathcal V$-值$\mathcal V$-态射的端可以使用超自然变换的更基本概念来定义。这反过来允许我们定义自然变换的\emph{对象},从而使我们能够定义加权极限。我们现在可以继续并扩展端的定义,以在$\mathcal V$-范畴$\mathcal D$中与混合变差值的更一般的$\mathcal V$-函子一起使用:
 \[ P \colon \mathcal C^{op} \otimes \mathcal C \to \mathcal D \]
 我们将使用此函子作为$\mathcal D$中的图表。

 在这一点上,数学家们开始担心规模问题。毕竟我们将整个范畴——平方——嵌入$\mathcal D$中的一个单一图表中。为了避免规模问题,我们只假设$\mathcal C$是小的;也就是说,它的对象形成一个集合。

 我们想对由$P$定义的图表采取加权极限。权重必须是一个$\mathcal V$-函子$\mathcal C^{op} \otimes \mathcal C \to \cat V$。我们有一个总是在我们手边的函子,同态函子$\text{Hom}_{\cat C}$。我们将使用它将端定义为加权极限:
 \[  \int_c P\langle c, c\rangle = \text{lim}^{\text{Hom}} P\]

 首先,让我们说服自己这个公式在普通($\Set$-丰盈)范畴中有效。由于端是通过其映射入属性定义的,让我们考虑从任意对象$d$到加权极限的映射,并使用标准的Yoneda技巧来证明同构。根据定义,我们有:
 \[ \cat D(d, \text{lim}^{\text{Hom}} P) \cong [\cat C^{op} \times \cat C, \Set](\cat C(-, =), \cat D (d, P(-, =))\]
 我们可以将自然变换的集合重写为在对象对$\langle c, c' \rangle$上的端:
 \[ \int_{\langle c, c' \rangle} \Set(\cat C(c, c'), \cat D (d, P\langle c, c' \rangle)) \]
 使用Fubini定理,这等效于迭代端:
 \[\int_c \int_{c'} \Set(\cat C(c, c'), \cat D(d, P\langle c, c' \rangle))\]
 我们现在可以应用ninja-Yoneda引理对$c'$进行积分。结果是:
 \[ \int_c \cat D(d, P\langle c, c \rangle) \cong \cat D(d, \int_c P \langle c, c \rangle) \]
 我们使用连续性将端推到同态函子下。由于$d$是任意的,我们得出结论,对于普通范畴:
 \[ \text{lim}^{\text{Hom}} P \cong  \int_c P\langle c, c\rangle \]
 这证明了我们使用加权极限在丰盈情况下定义端是合理的。

 类似的公式适用于余端,除了我们使用对立范畴中的同态函子$\text{Hom}_{\cat C^{op}}$作为权重的余极限:
 \[  \int^c P\langle c, c\rangle = \text{colim}^{\text{Hom}_{\cat C^{op}}} P\]

 \begin{exercise}
  证明对于普通的$\Set$-丰盈范畴,余端的加权余极限定义再现了先前的定义。提示:使用余端的映射出属性。
 \end{exercise}

 \begin{exercise}
  证明,只要双方存在,以下等式在普通($\Set$-丰盈)范畴中成立(可以推广到丰盈环境):
  \[ \text{lim}^W D \cong \int_{j \colon \cat J} W j \pitchfork D j \]
  \[ \text{colim}^W D \cong \int^{j \colon \cat J} W j \cdot D j \]
  提示:使用Yoneda技巧和幂与共幂的定义进行映射入/出。
 \end{exercise}

 \section{Kan Extensions(Kan扩张)}

 我们已经看到如何使用来自奇点范畴$1$的函子将极限和余极限表达为Kan扩张。加权极限让我们摆脱了奇点,明智地选择权重使我们能够将Kan扩张表达为加权极限。

 首先,让我们推导普通的$\Set$-丰盈范畴的公式。右Kan扩张定义为:
 \[ (\text{Ran}_P F) e \cong \int_c \cat B (e, P c) \pitchfork F c \]
 我们将考虑从任意对象$d$到它的映射。推导遵循几个简单的步骤,主要是通过扩展定义。

 我们从以下开始:
 \[ \cat D\big(d, (\text{Ran}_P F) e\big) \]
 并代入Kan扩张的定义:
 \[ \cat D\big(d, \int_c \cat B (e, P c) \pitchfork F c\big) \]
 使用同态函子的连续性,我们可以将端拉出:
 \[ \int_c \cat D\big(d,  \cat B (e, P c) \pitchfork F c\big) \]
 然后我们使用pitchfork的定义:
 \[ \cat D (d, A \pitchfork d') \cong \Set  \big(A, \cat D(d, d')\big) \]
 得到:
 \[ \int_c \cat D\big(d,  \cat B (e, P c) \pitchfork F c\big) \cong \int_c \Set\big( \cat B(e, P c), \cat D(d, F c)\big)\]
 这可以写成一个自然变换的集合:
 \[ [\cat C, \Set] \big(\cat B(e, P-), \cat D(d, F-)\big) \]
 加权极限也通过自然变换的集合来定义:
 \[  \mathcal D(d, \text{lim}^W F) \cong [\mathcal C, \Set] \big(W, \mathcal D(d, F-)\big) \]
 我们得出最终结果:
 \[ \cat D(d, \text{lim}^{\cat B (e, P-)} F) \]
 由于$d$是任意的,我们可以使用Yoneda技巧得出结论:
 \[ (\text{Ran}_P F) e = \text{lim}^{\cat B(e, P-)} F  \]
 这个公式成为丰盈环境中右Kan扩张的定义。

 类似地,左Kan扩张可以定义为加权余极限:
 \[ (\text{Lan}_P F) e = \text{colim}^{\cat B(P-, e)} F \]

 \begin{exercise}
  推导普通范畴左Kan扩张的公式。
 \end{exercise}

 \section{Useful Formulas(有用公式)}
 \begin{itemize}
  \item Yoneda引理:
  \[ \int_x [\mathcal C( c, x), F x] \cong F c \]
  \item 加权极限:
  \[  \mathcal C(x, \text{lim}^W D) \cong [\mathcal J, \mathcal V] \big(W, \mathcal C(x, D-)\big) \]
  \item 加权余极限:
  \[  \mathcal C(\text{colim}^W D, x) \cong [\mathcal J^{op}, \mathcal V] \big(W, \mathcal C(D-, x)\big) \]
  \item 右Kan扩张:
  \[ (\text{Ran}_P F) e = \text{lim}^{\cat B(e, P-)} F  \]
  \item 左Kan扩张:
  \[ (\text{Lan}_P F) e = \text{colim}^{\cat B(P-, e)} F \]
 \end{itemize}

\end{document}
