\documentclass[DaoFP]{subfiles}
\begin{document}
 \setcounter{chapter}{3}

 \chapter{Sum Types\\和类型}

 \section{Bool\\布尔类型}

 我们已经知道如何组合箭头。那么我们如何组合对象呢?

 我们已经定义了$0$(初始对象)和$1$(终端对象)。那么$2$是什么?它就是$1$加$1$。

 一个$2$是一个具有两个元素的对象:从$1$发出的两个箭头。我们将其中一个箭头称为\hask{True},另一个箭头称为\hask{False}。不要将这些名称与初始和终端对象的逻辑解释混淆。这两个只是\emph{箭头}。

 \[
  \begin{tikzcd}
   1
   \arrow[dd, bend right, "\text{True}"']
   \arrow[dd, bend left, "\text{False}"]
   \\
   \\
   2
  \end{tikzcd}
 \]

 这个简单的概念可以立即用Haskell表达\footnote{这种定义风格在Haskell中称为广义代数数据类型或\hask{GADTs}},即定义一个类型,传统上称为\hask{Bool},以纪念其发明者George Boole(1815-1864)。

 \begin{haskell}
  data Bool where
  True  :: () -> Bool
  False :: () -> Bool
 \end{haskell}
 它对应于相同的图(只是在Haskell中重命名了一些内容):
 \[
  \begin{tikzcd}
   \hask{()}
   \arrow[dd, bend right, "\hask{True}"']
   \arrow[dd, bend left, "\hask{False}"]
   \\
   \\
   \hask{Bool}
  \end{tikzcd}
 \]

 正如我们之前所见,元素有一种简写表示法,因此这是一个更紧凑的版本:

 \begin{haskell}
  data Bool where
  True  :: Bool
  False :: Bool
 \end{haskell}

 现在我们可以定义\hask{Bool}类型的一个项,例如
 \begin{haskell}
  x :: Bool
  x = True
 \end{haskell}
 第一行声明\hask{x}是\hask{Bool}的一个元素(实际上是一个函数\hask{()->Bool}),第二行告诉我们它是两个元素中的哪个。

 我们在\hask{Bool}定义中使用的\hask{True}和\hask{False}函数被称为\emph{数据构造器}。它们可以用来构造特定的项,就像上面的例子一样。顺便提一下,在Haskell中,函数名称通常以小写字母开头,除非它们是数据构造器。

 我们对\hask{Bool}类型的定义仍然不完整。我们知道如何构造一个\hask{Bool}项,但我们不知道如何使用它。我们必须能够定义从\hask{Bool}发出的箭头——即从\hask{Bool}到其他类型的\emph{映射}。

 第一个观察结果是,如果我们有一个从\hask{Bool}到某个具体类型\hask{A}的箭头\hask{h},那么我们自动会得到两个从单位对象到\hask{A}的箭头,只需通过组合即可。以下两个(扭曲的)三角形是可交换的:

 \[
  \begin{tikzcd}
   \hask{()}
   \arrow[dd, bend right, "\text{True}"']
   \arrow[dd, bend left, "\text{False}"]
   \arrow[ddd, bend right =90, "x"']
   \arrow[ddd, bend left=90, "y"]
   \\
   \\
   \text{Bool}
   \arrow[d, dashed, "h"]
   \\
   A
  \end{tikzcd}
 \]
 换句话说,每个\hask{Bool->A}函数都会生成一对\hask{A}的元素。

 给定一个具体类型\hask{A}:
 \begin{haskell}
  h :: Bool -> A
 \end{haskell}
 我们有:
 \begin{haskell}
  x = h True
  y = h False
 \end{haskell}
 其中
 \begin{haskell}
  x :: A
  y :: A
 \end{haskell}
 注意,这里使用了应用函数到元素的简写表示法:
 \begin{haskell}
  h True -- 意味着: h . True
 \end{haskell}

 我们现在准备通过添加一个条件来完成对\hask{Bool}的定义,即从\hask{Bool}到\hask{A}的任何函数不仅生成,而且\emph{等价于}一对\hask{A}的元素。换句话说,一对元素唯一地决定了从\hask{Bool}的函数。

 这意味着我们可以以两种方式解释上面的图:给定\hask{h},我们可以很容易地得到\hask{x}和\hask{y}。但反过来也成立:一对元素\hask{x}和\hask{y}唯一地\emph{定义}了\hask{h}。

 这里有一个双射在起作用。这次是元素对$(x, y)$与箭头$h$之间的一对一映射。

 在Haskell中,\hask{h}的定义封装在\hask{if}, \hask{then}, \hask{else}结构中。给定
 \begin{haskell}
  x :: A
  y :: A
 \end{haskell}
 我们定义映射为
 \begin{haskell}
  h :: Bool -> A
  h b = if b then x else y
 \end{haskell}
 在这里,\hask{b}是\hask{Bool}类型的一个项。

 一般来说,数据类型是通过\emph{引入规则}创建的,并通过\emph{消除规则}解构。 \hask{Bool}数据类型有两个引入规则,一个使用\hask{True},另一个使用\hask{False}。 \hask{if}, \hask{then}, \hask{else}结构定义了消除规则。

 考虑到上面对\hask{h}的定义,我们可以检索用于定义它的两个项,这被称为\emph{计算规则}。它告诉我们如何计算\hask{h}的结果。如果我们用\hask{True}调用\hask{h},结果是\hask{x};如果我们用\hask{False}调用它,结果是\hask{y}。

 我们永远不要忘记编程的目的:将复杂问题分解为一系列更简单的问题。 \hask{Bool}的定义说明了这个想法。每当我们需要构造一个从\hask{Bool}到某类型的映射时,我们将其分解为构造目标类型的一对元素的两个更小的任务。我们用两个更简单的问题换取了一个更大的问题。

 \section{Enumerations\\枚举类型}

 在0、1和2之后是什么?一个具有三个数据构造器的对象。例如:
 \begin{haskell}
  data RGB where
  Red   :: RGB
  Green :: RGB
  Blue  :: RGB
 \end{haskell}
 如果您厌倦了冗长的语法,可以使用此类型定义的简写:

 \begin{haskell}
  data RGB = Red | Green | Blue
 \end{haskell}
 此引入规则允许我们构造\hask{RGB}类型的项,例如:
 \begin{haskell}
  c :: RGB
  c = Blue
 \end{haskell}
 要定义从\hask{RGB}到其他类型的映射,我们需要更通用的消除模式。就像从\hask{Bool}到某个类型的函数由两个元素决定一样,从\hask{RGB}到\hask{A}的函数由\hask{A}的三个元素\hask{x}, \hask{y}, \hask{z}决定。我们使用模式匹配语法来编写这样的函数:
 \begin{haskell}
  h :: RGB -> A
  h Red   = x
  h Green = y
  h Blue  = z
 \end{haskell}
 这只是一个函数,它的定义分为三种情况。

 我们也可以使用相同的语法用于\hask{Bool},代替\hask{if}, \hask{then}, \hask{else}:
 \begin{haskell}
  h :: Bool -> A
  h True  = x
  h False = y
 \end{haskell}
 实际上,还有第三种编写相同内容的方法,即\hask{case}语句:
 \begin{haskell}
  h c = case c of
  Red   -> x
  Green -> y
  Blue  -> z
 \end{haskell}
 甚至是:
 \begin{haskell}
  h :: Bool -> A
  h b = case b of
  True  -> x
  False -> y
 \end{haskell}
 编程时,您可以根据需要使用这些方式中的任何一种。

 这些模式也适用于具有四个、五个或更多数据构造器的类型。例如,十进制数字是以下之一:
 \begin{haskell}
  data Digit = Zero | One | Two | Three | ... | Nine
 \end{haskell}

 有一个巨大的Unicode字符枚举,称为\hask{Char}。它们的构造器有特殊的名称:你在字符本身前后加上单引号,例如:
 \begin{haskell}
  c :: Char
  c = 'a'
 \end{haskell}

 正如老子所说,万物之模式,需时多年,故人们发明了通配符模式“下划线”,它匹配一切。

 由于模式是按顺序匹配的,因此您应该将通配符模式作为系列中的最后一个:
 \begin{haskell}
  yesno :: Char -> Bool
  yesno c = case c of
  'y' -> True
  'Y' -> True
  _   -> False
 \end{haskell}

 但我们为什么要止步于此呢?类型\hask{Int}可以被看作是在$-2^{29}$到$2^{29}$范围内的整数枚举(或更多,取决于实现)。当然,在这种范围内进行详尽的模式匹配是不可行的,但原理仍然成立。

 实际上,类型\hask{Char}用于Unicode字符,\hask{Int}用于固定精度整数,\hask{Double}用于双精度浮点数,以及其他几个类型,都内置于语言中。

 这些不是无限类型。它们的元素可以被枚举,即使可能需要一万年。类型\hask{Integer}则是无限的。

 \section{Sum Types\\和类型}

 \hask{Bool}类型可以看作是和类型$2 = 1 + 1$。但没有什么能阻止我们用其他类型替换$1$,甚至用不同的类型替换这两个$1$。我们可以使用两个箭头定义一个新类型$a + b$。我们称它们为\hask{Left}和\hask{Right}。定义图就是引入规则:

 \[
  \begin{tikzcd}
   a
   \arrow[dr,  "\text{Left}"']
   && b
   \arrow[dl, "\text{Right}"]
   \\
   &a + b
  \end{tikzcd}
 \]

 在Haskell中,类型$a + b$被称为\hask{Either a b}。类似于\hask{Bool},我们可以将其定义为:
 \begin{haskell}
  data Either a b where
  Left  :: a -> Either a b
  Right :: b -> Either a b
 \end{haskell}
 (注意小写字母用于类型变量。)

 类似地,从$a + b$到某个类型$c$的映射由以下可交换图决定:
 \[
  \begin{tikzcd}
   a
   \arrow[dr,  bend left, "\text{Left}"']
   \arrow[ddr, bend right, "f"']
   && b
   \arrow[dl, bend right, "\text{Right}"]
   \arrow[ddl, bend left, "g"]
   \\
   &a + b
   \arrow[d, dashed, "h"]
   \\
   & c
  \end{tikzcd}
 \]

 给定一个函数$h$,我们只需通过组合它与\hask{Left}和\hask{Right}即可得到一对函数$f$和$g$。反过来,这对函数唯一地决定了$h$。这是消除规则。

 当我们想将此图转换为Haskell时,我们需要选择两种类型的元素。我们可以通过定义从终端对象的箭头来完成:

 \[
  \begin{tikzcd}
   &1
   \arrow[ld, "a"']
   \arrow[rd, "b"]
   \\
   a
   \arrow[dr,  bend left, "\text{Left}"']
   \arrow[ddr, bend right, "f"']
   && b
   \arrow[dl, bend right, "\text{Right}"]
   \arrow[ddl, bend left, "g"]
   \\
   &a + b
   \arrow[d, dashed, "h"]
   \\
   & c
  \end{tikzcd}
 \]

 沿着这个图的箭头,我们得到:
 \[h \circ \text{Left} \circ a = f \circ a\]
 \[h \circ \text{Right} \circ b = g \circ b\]

 Haskell语法几乎逐字重复了这些方程,最终得到这个模式匹配语法用于定义\hask{h}:

 \begin{haskell}
  h :: Either a b -> c
  h (Left  a) = f a
  h (Right b) = g b
 \end{haskell}
 (再次注意小写字母用于类型变量,并且这些字母与这些类型的项相同。与人类不同,编译器不会因此感到困惑。)

 你还可以从右向左阅读这些方程,这样你就会看到和类型的计算规则:定义\hask{h}所使用的两个函数可以通过应用\hask{h}到\hask{(Left a)}和\hask{(Right b)}来恢复。

 你也可以使用\hask{case}语法定义\hask{h}:
 \begin{haskell}
  h e = case e of
  Left  a -> f a
  Right b -> g b
 \end{haskell}

 那么,数据类型的本质是什么?不过是操纵箭头的一个方法。

 \section{Maybe\\可能类型}

 一个非常有用的数据类型,\hask{Maybe}被定义为$1 + a$,对于任意$a$。这是它在Haskell中的定义:
 \begin{haskell}
  data Maybe a where
  Nothing :: () -> Maybe a
  Just    ::  a -> Maybe a
 \end{haskell}
 数据构造器\hask{Nothing}是从单位类型的箭头,而\hask{Just}从\hask{a}构造\hask{Maybe a}。 \hask{Maybe a}与\hask{Either () a}是同构的。它也可以用简写符号定义:
 \begin{haskell}
  data Maybe a = Nothing | Just a
 \end{haskell}

 \hask{Maybe}主要用于对部分函数的返回类型进行编码:这些函数在某些参数值上未定义。在这种情况下,这些函数不会失败,而是返回\hask{Nothing}。在其他编程语言中,部分函数通常使用异常(或核心转储)来实现。

 \section{Logic\\逻辑}

 在逻辑中,命题$A + B$被称为\emph{逻辑或}。你可以通过提供$A$的证明或$B$的证明来证明它。任意一个证明都足够了。

 如果你想证明$C$从$A+B$中得出,你必须为两种可能性做好准备:一种是有人通过证明$A$来证明$A+B$(此时$B$可能是假的),另一种是通过证明$B$来证明$A+B$(此时$A$可能是假的)。在第一种情况下,你必须展示$C$如何从$A$中得出。在第二种情况下,你需要证明$C$从$B$中得出。这些正是$A+B$消除规则中的箭头。

 \section{Cocartesian Categories\\余笛卡尔范畴}

 在Haskell中,我们可以使用\hask{Either}定义任意两种类型的和。一个类别中,如果所有和存在,并且初始对象也存在,我们称这个类别为\emph{余笛卡尔},而和则被称为\emph{余积}。你可能注意到,和类型模仿了数字的加法。事实证明,初始对象扮演了零的角色。

 \subsection{One Plus Zero\\一加零}

 我们首先展示$1 + 0 \cong 1$,这意味着终端对象与初始对象的和是同构于终端对象的。此类证明的标准程序是使用Yoneda技巧。由于和类型是通过映射出去定义的,我们应该比较它们的箭头。

 Yoneda论证说,如果两个对象有一个自然的双射$\beta_a$,它在从它们到任意对象$a$的箭头集合之间建立,那么这两个对象是同构的。

 让我们看看$1 + 0$的定义以及它到任意对象$a$的映射。这个映射由一对$(x, \mbox{!`})$定义,其中$x$是$a$的一个元素,$\mbox{!`}$是从初始对象到$a$的唯一箭头(在Haskell中是\hask{absurd}函数)。

 \[
  \begin{tikzcd}
   1
   \arrow[dr,  bend left, "\text{Left}"']
   \arrow[ddr, bend right, "x"']
   && 0
   \arrow[dl, bend right, "\text{Right}"]
   \arrow[ddl, bend left, "\mbox{!`}"]
   \\
   &1 + 0
   \arrow[d, dashed, "h"]
   \\
   & a
  \end{tikzcd}
  \qquad
  \begin{tikzcd}
   1
   \arrow[dd, "x"]
   \\
   \\
   a
  \end{tikzcd}
 \]

 我们想在从$1+0$发出的箭头与从$1$发出的箭头之间建立一对一的映射。箭头$h$由一对$(x, \mbox{!`})$决定。由于只有一个$\mbox{!`}$,因此$h$与$x$之间存在一个双射。

 我们定义$\beta_a$将由一对$(x, \mbox{!`})$定义的任何$h$映射到$x$。反过来,$\beta^{-1}_a$将$x$映射到一对$(x, \mbox{!`})$。但这是一个自然变换吗?

 要回答这个问题,我们需要考虑当我们从$a$切换到通过箭头$g \colon a \to b$与$a$连接的某个$b$时会发生什么。我们现在有两个选项:
 \begin{itemize}
  \item 使$h$通过后组合$x$和$\mbox{!`}$与$g$来切换焦点。我们得到一个新对$(y = g \circ x, \mbox{!`})$。然后使用$\beta_b$。
  \item 使用$\beta_a$将$(x, \mbox{!`})$映射到$x$。然后跟随$(g \circ -)$进行后组合。
 \end{itemize}
 在这两种情况下,我们都得到相同的箭头$y = g \circ x$。因此映射$\beta$是自然的。因此$1 + 0$与$1$是同构的。

 在Haskell中,我们可以定义构成同构的两个函数,但没有直接表达它们是彼此逆的方式。
 \begin{haskell}
  f :: Either () Void -> ()
  f (Left ()) = ()
  f (Right _) = ()

  f_1 :: () -> Either () Void
  f_1 _ = Left ()
 \end{haskell}
 函数定义中的下划线通配符表示该参数被忽略。函数\hask{f}的第二个子句是多余的,因为\hask{Void}类型没有项。

 \subsection{Something Plus Zero\\某物加零}

 一个非常相似的论证可以用来证明$a + 0 \cong a$。以下图解释了这一点。

 \[
  \begin{tikzcd}
   a
   \arrow[dr,  bend left, "\text{Left}"']
   \arrow[ddr, bend right, "f"']
   && 0
   \arrow[dl, bend right, "\text{Right}"]
   \arrow[ddl, bend left, "\mbox{!`}"]
   \\
   &a + 0
   \arrow[d, dashed, "h"]
   \\
   & x
  \end{tikzcd}
  \qquad
  \begin{tikzcd}
   a
   \arrow[dd, "f"]
   \\
   \\
   x
  \end{tikzcd}
 \]

 我们可以通过实现一个(多态)函数\hask{h},它适用于任何类型\hask{a},将此论证转化为Haskell。

 \begin{exercise}
  在Haskell中实现两个函数,这两个函数形成了\hask{(Either a Void)}与\hask{a}之间的同构。
 \end{exercise}

 我们可以使用类似的论证来证明$0 + a \cong a$,但和类型的更一般属性使得这点显得多余。
 \subsection{Commutativity\\交换性}

 和类型定义中的图具有漂亮的左右对称性,这表明它满足交换规则,即$a + b \cong b + a$。

 让我们考虑公式两边的映射。你可以很容易地看到,对于左侧由一对$(f, g)$决定的每个$h$,右侧有一个对应的$h'$,它由一对$(g, f)$给出。这就建立了箭头之间的双射。

 \[
  \begin{tikzcd}
   a
   \arrow[dr,  bend left, "\text{Left}"']
   \arrow[ddr, bend right, "f"']
   && b
   \arrow[dl, bend right, "\text{Right}"]
   \arrow[ddl, bend left, "g"]
   \\
   &a + b
   \arrow[d, dashed, "h"]
   \\
   & x
  \end{tikzcd}
  \qquad
  \begin{tikzcd}
   b
   \arrow[dr,  bend left, "\text{Left}"']
   \arrow[ddr, bend right, "g"']
   && a
   \arrow[dl, bend right, "\text{Right}"]
   \arrow[ddl, bend left, "f"]
   \\
   &b + a
   \arrow[d, dashed, "h'"]
   \\
   & x
  \end{tikzcd}
 \]

 \begin{exercise}
  证明上面定义的双射是自然的。提示:$f$和$g$通过与$k \colon x \to y$的后组合来改变焦点。
 \end{exercise}

 \begin{exercise}
  在Haskell中实现见证\hask{(Either a b)}与\hask{(Either b a)}之间同构的函数。注意,这个函数本身是它的逆。
 \end{exercise}

 \subsection{Associativity\\结合性}

 正如在算术中,我们定义的和是结合的:
 \[(a + b) + c \cong a + (b + c)\]
 编写左侧的映射很容易:
 \begin{haskell}
  h :: Either (Either a b) c -> x
  h (Left (Left a))  = f1 a
  h (Left (Right b)) = f2 b
  h (Right c)        = f3 c
 \end{haskell}
 注意使用嵌套模式,如\hask{(Left (Left a))}等。该映射完全由三个函数定义。可以使用相同的函数来定义右侧的映射:
 \begin{haskell}
  h' :: Either a (Either b c) -> x
  h' (Left a)          = f1 a
  h' (Right (Left b))  = f2 b
  h' (Right (Right c)) = f3 c
 \end{haskell}
 这建立了两个定义映射的函数组之间的一对一映射。这种映射是自然的,因为所有的焦点转换都是通过后组合完成的。因此,两边是同构的。

 这个代码也可以用图形化的形式展示。以下是同构左侧的图:

 \[
  \begin{tikzcd}
   a
   \arrow[rd, "L"']
   \arrow[rrddd, bend right, red, "f_1"']
   && b
   \arrow[ld, "R"]
   \arrow[ddd, bend left=60, red, "f_2"]
   &&c
   \arrow[lldd, "R"]
   \arrow[llddd, bend left, red, "f_3"]
   \\
   & a + b
   \arrow[rd, "L"']
   \\
   &&(a + b) + c
   \arrow[d, dashed, "h"]
   \\
   && x
  \end{tikzcd}
 \]

 \subsection{Functoriality\\函子性}

 由于和是由映射出的性质定义的,所以很容易看出当我们改变焦点时会发生什么:它会随着定义积的箭头焦点的变化而“自然地”变化。但当我们移动这些箭头的源时,会发生什么呢?

 假设我们有箭头,将$a$和$b$映射到某个$a'$和$b'$:
 \begin{align*}f &\colon a \to a' \\
 g &\colon b \to b'
 \end{align*}
 这些箭头与构造器$\text{Left}$和$\text{Right}$分别组合,定义了和之间的映射:

 \[
  \begin{tikzcd}
   a
   \arrow[d, "f"]
   \arrow[dr,  bend left, "\text{Left}"']
   && b
   \arrow[d, "g"]
   \arrow[dl, bend right, "\text{Right}"]
   \\
   a'
   \arrow[rd, "\text{Left}"']
   &a + b
   \arrow[d, dashed, "h"]
   & b'
   \arrow[ld, "\text{Right}"]
   \\
   & a' + b'
  \end{tikzcd}
 \]
 箭头对$(\text{Left} \circ f, \text{Right} \circ g)$唯一地定义了箭头$h \colon a + b \to a' + b'$。

 这种和的性质称为\emph{函子性}。你可以想象它允许你在和的内部转换两个对象,并获得一个新的和。我们还说函子性允许我们\emph{提升}一对箭头以操作和。

 \begin{exercise}
  证明函子性保持组合。提示:取两条可组合的箭头,$g \colon b \to b'$和$g' \colon b' \to b''$,证明应用$g' \circ g$给出与首先应用$g$将$a + b$变换为$a + b'$,然后应用$g'$将$a + b'$变换为$a + b''$相同的结果。
 \end{exercise}

 \begin{exercise}
  证明函子性保持身份。提示:使用$id_b$并证明它被映射到$id_{a+b}$。
 \end{exercise}

 \subsection{Symmetric Monoidal Category\\对称幺半范畴}

 当一个孩子学习加法时,我们称之为算术。当一个成年人学习加法时,我们称之为余笛卡尔范畴。

 无论我们是加数字,组合箭头,还是构造对象的和,我们都在重复使用将复杂事物分解为其简单组成部分的相同思想。

 正如老子所说,当事物结合形成新事物时,如果操作是结合的,并且有一个中性元素,我们就知道如何处理万事万物。

 我们定义的和类型满足这些属性:
 \begin{align*}
  a + 0 &\cong a \\
  a + b &\cong b + a \\
  (a + b) + c &\cong a + (b + c)
 \end{align*}
 并且它是函子性的。具有这种操作的范畴称为\emph{对称幺半范畴}。当操作是和(余积)时,它被称为\emph{余笛卡尔}。在下一章中,我们将看到另一种被称为\emph{笛卡尔}的幺半结构,不带“余”字。
\end{document}
